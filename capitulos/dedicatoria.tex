
\begin{dedicatoria}
	\vspace*{\fill}
	\centering
	\noindent
	\textit{Para a gata Eleonora} \vspace*{\fill}
\end{dedicatoria}
% ---

% ---
% Agradecimentos
% ---
\begin{agradecimentos}

%Felizmente não falta gente a ser agradecida aqui. Da família, devo enormemente à Joana, José Carlos, Nordeval, Carmosa, Maria Helena, Ivelene e Luiza - sem os quais nem aqui estaria. Todos vocês me deram muito, muito amor.

%Dos amigos que fizeram a sábia escolha de não partilhar uma graduação comigo agradeço enormemente Rômulo, Lucas, Leonardo Cesar, Leonardo Martinelli, Maurício, Carolina, Marcelo, João Victor e Mauro. Obrigado pelo apoio, pelas noites e pelo carinho, tenho cada um de vocês em um lugar especial. Dos que trilharam e trilham este caminho profissional comigo, devo muito a Jamil Civitarese, que me ensinou muito mais do que econometria, Daniel Duque, que entre o início e o fim desta graduação me deu votos de confiança fundamentais  e se tornou uma sã e necessária figura pública. Flavio Abdenur e Diego Cardoso são pessoas incríveis e para mim foram como amigáveis faróis, sempre animados em compartilhar dois centavos de experiência. Não sei dizer o que seria de mim sem Daniel Coutinho e seus aparentemente infinitos saberes. Alexandre Portugal, Leonardo Grandelle e Bruno Lopes foram presentes da UFF e a eles devo alguns dos momentos mais felizes destes últimos quatro anos.

%Não foram poucos mestres nestes 22 anos de vida. O incrível casal Ludmilla & José Nazareno, Diomário Silva, Claudio França e João Aprígio foram professores de escola que guardo no coração. Bruno Santiago me ajudou a trilhar o sinuoso e surpreendentemente prazeroso caminho da Matemática, sempre com um sorriso no rosto e uma paciência invejável. Ele é um dos grandes responsáveis (ou devo dizer culpados?) pela direção em que hoje ando. Devo muito às suas aulas, cafés e conselhos. Jesus Obregon é talvez o maior responsável pela conclusão de graduação que se aproxima e a ele agradeço cada um dos seguidos e significantes votos de confiança. Wilson Calmon é um gigante em cujos ombros sou muito feliz de ter tido a oportunidade de me apoiar. Luciano Vereda foi um excelente professor, chegou perigosamente próximo de me convencer que Macroeconomia é uma área interessante com algumas das melhores aulas que tive e provavelmente terei na vida. Também agradeço Paulo Gusmão, Juliana Coelho e Jony Arrais, todos professores cuidadosos, diligentes e compreensivos com este aspirante a Economista se aventurando pela Matemática e Estatística.

%Tive o prazer e a sorte de ter muitas experiências fora do campus da universidade nestes últimos anos. Agradeço muito à equipe do Instituto Mercado Popular, onde dei os primeiros passos no mundo da análise e apresentação de dados - especialmente a Carlos Góes, André Spigariol, Luan Sperandio, Pedro Menezes, Diego Cardoso e Flavio Abdenur. A Flavio e Diego em particular agradeço pelas incontáveis particulares. No Instituto de Pesquisa Econômica Aplicada do Rio de Janeiro, Felipe Russo, Maíra Franca e Carlos Henrique Corseiul. Na Fundação Getúlio Vargas, Jamil Civitarese e Amanda Medeiros. Na Análise Macro, Vitor Wilher. Na Mutual, Patrick Maia.

%Agradeço, por fim, a todos e todas cujos nomes fatalmente não foram citados, mas que participaram de alguma forma. Obrigado por tudo. 


\end{agradecimentos}