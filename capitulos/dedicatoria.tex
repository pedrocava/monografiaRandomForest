
\begin{agradecimentos}

Tive o enorme privilégio de crescer em um lar cheio de amor e dedicação. É algo que a passagem do tempo deixa continuamente mais claro. Devo tudo à minha (grande) família e suas figuras curiosas. Luiza, Carmosa, José Carlos, Maria Helena, Joana e outros tantos nomes queridos são figuras essenciais para mim.

Aos amigos, agradeço profundamente pela presença. Leonardo(s), Mauro, Rafael, Marcelo(s), João Pedro, Maurício, Rômulo, Ana Luiza, José Victor, Lucas, Carolina, Caetano e tantos outros que fatalmente não citarei. Obrigado por tudo.

Devo muito aos colegas de profissão. Jamil Civitarese, Tiago Dantas, Filipe Russo, Maíra Franca, Armando Martins, Flavio Abdenur, Diego Cardoso foram figuras importantes, sempre dispostas a compartilhar alguns centavos de experiência. Patrick Maia, um chefe-colega que muito me ensinou. Daniel Duque, especialmente, por ter acompanhado esses anos de graduação com um cuidado muito bonito. Mais ainda foi vê-lo se tornar uma muito necessária figura pública.

Aos mestres, agradeço muito a Paulo Gusmão, Ana Urraca, Juliana Coelho, Wilson Calmon, Diogo Bravo e Carlos Gabriel Guimarães. Devo agradecimentos especiais aos professores Jesus Alexei Luizar Obregon, que não sabe disso mas foi um dos poucos motivos que me impediram de trocar de graduação em certo momento, Luciano Vereda que quase me convenceu de que macroeconomia é um campo interessante e Bruno Santiago que com um sorriso, um café e paciência aparentemente infinita me ajudou a dar os doloridos passos iniciais na matemática de gente grande.

Por fim, agradeço ao Instituto de Pesquisa Econômica Aplicada (IPEA) do Rio de Janeiro, à Escola Brasileira de Administração Pública e Privada (EBAPE/FGV) e à Análise Macro pelas oportunidades de estágio e aprendizado. Também listo a Genesis Library, que viabilizou os meus estudos em vários momentos chave. 

A todos que participaram e não foram citados, agradeço enormemente. 





\end{agradecimentos}