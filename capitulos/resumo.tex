% ------------------------------------------------------------------
% RESUMO E ABSTRACT
%
% Para iniciarmos um ambiente, seja de resumo, abstract ou outro 
% qualquer, devemos começar pelo \begin. Onde eu escrevo nesse 
% arquivo, já me encontro dentro de um ``ambiente de resumo''.
%
% Aqui, apresentarei o resumo e mais um outro comando de exemplo: o 
% \lipsum. A principal função do \lipsum é gerar textos aleatórios -
% textos dummys. Assim, por mais que no presente documento - 
% no arquivo .TeX - os \lipsum[3-5] ou \lipsum[2-4] apareçam, no 
% .pdf esses só aparecerão como textos aleatórios.
% ------------------------------------------------------------------

%\setlength{\absparsep}{18pt} % ajusta o espaçamento dos parágrafos do resumo
\begin{resumo}

Este trabalho demonstra a construção partindo de primeiros princípios de modelos de floresta aleatória, os compara com modelos lineares da Econometria Clássica e expõe a problemática de recuperar os efeitos marginais. Um algoritmo de computação dos efeitos marginais a partir de uma floresta aleatória treinada, uma simulação de Monte Carlo mostrando em ambiente controlado sua aplicação e uma aplicação prática com dados de preços de imóveis são apresentados. Por fim, agendas futuras de pesquisa são esboçadas.


\textbf{Palavras-chave}: Florestas Aleatórias; Aprendizado de Máquina; Econometria. % Palavra-chave inicia-se com maiúscula
\end{resumo}

% ------------------------------------------------------------------
% No caso do abstract, faremos a mesma coisa. Só adicionaremos a 
% opção abstract como o argumento do comando.
%-------------------------------------------------------------------

\begin{resumo}[ABSTRACT] % ESCREVER EM LETRAS MAIÚSCULAS

This work demonstrates the construction starting from the first principles of random forest models, compares them with linear models of Classical Econometrics and exposes the problem of recovering marginal effects. An algorithm for computing the marginal effects from a trained random forest, a Monte Carlo simulation showing its application in a controlled environment and a practical application with real estate price data are presented. Finally, future research agendas are outlined.

\textbf{Keywords}: Random Forests; Machine Learning; Econometrics. 	
\end{resumo}



