% ------------------------------------------------------------------
% RESUMO E ABSTRACT
%
% Para iniciarmos um ambiente, seja de resumo, abstract ou outro 
% qualquer, devemos começar pelo \begin. Onde eu escrevo nesse 
% arquivo, já me encontro dentro de um ``ambiente de resumo''.
%
% Aqui, apresentarei o resumo e mais um outro comando de exemplo: o 
% \lipsum. A principal função do \lipsum é gerar textos aleatórios -
% textos dummys. Assim, por mais que no presente documento - 
% no arquivo .TeX - os \lipsum[3-5] ou \lipsum[2-4] apareçam, no 
% .pdf esses só aparecerão como textos aleatórios.
% ------------------------------------------------------------------

%\setlength{\absparsep}{18pt} % ajusta o espaçamento dos parágrafos do resumo
\begin{resumo}

Floresta Aleatória é um modelo de classificação estatística construído agregando de árvores de decisão, cujas aplicações práticas vão de previsão de risco de crédito e preços de ativos, à avaliação de políticas públicas, reconhecimento de padrões visuais, prevenção contra fraudes e detecção de doenças em exames médicos. Embora de uso amplo e disseminado, há pouca literatura documentando as relações sistemáticas entre variações nas variáveis explicativas e a correspondente na variável de resposta. Primeiro a teoria principal por trás da técnica é exposta, então a problemática da estimação de efeitos marginais é apresentada, uma formulação matemática e analítica do problema é encontrada e um procedimento prático de estimação e cálculo dos efeitos marginais baseado em simulações de Monte Carlo é apresentado. Por fim, um estudo de caso com dados de acidentes em rodovias federais da Polícia Rodoviária Federal estima o "Efeito Bolsonaro". 


\textbf{Palavras-chave}: Florestas Aleatórias; Classificação Estatística; Aprendizado de Máquina; Econometria Aplicada. % Palavra-chave inicia-se com maiúscula
\end{resumo}

% ------------------------------------------------------------------
% No caso do abstract, faremos a mesma coisa. Só adicionaremos a 
% opção abstract como o argumento do comando.
%-------------------------------------------------------------------

\begin{resumo}[ABSTRACT] % ESCREVER EM LETRAS MAIÚSCULAS
	


\textbf{Keywords}: Latex; Abntex; Lipsum; Economics. 	
\end{resumo}



