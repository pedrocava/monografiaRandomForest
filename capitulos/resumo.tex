% ------------------------------------------------------------------
% RESUMO E ABSTRACT
%
% Para iniciarmos um ambiente, seja de resumo, abstract ou outro 
% qualquer, devemos começar pelo \begin. Onde eu escrevo nesse 
% arquivo, já me encontro dentro de um ``ambiente de resumo''.
%
% Aqui, apresentarei o resumo e mais um outro comando de exemplo: o 
% \lipsum. A principal função do \lipsum é gerar textos aleatórios -
% textos dummys. Assim, por mais que no presente documento - 
% no arquivo .TeX - os \lipsum[3-5] ou \lipsum[2-4] apareçam, no 
% .pdf esses só aparecerão como textos aleatórios.
% ------------------------------------------------------------------

%\setlength{\absparsep}{18pt} % ajusta o espaçamento dos parágrafos do resumo
\begin{resumo}

Floresta Aleatória é um modelo estatístico de aprendizado supervisionado construído agregando de árvores de decisão, cujas aplicações práticas vão de previsão de risco de crédito e preços de ativos, à avaliação de políticas públicas, reconhecimento de padrões visuais, prevenção contra fraudes e detecção de doenças em exames médicos. Embora de uso amplo e disseminado, há pouca literatura documentando as relações sistemáticas entre variações nas variáveis explicativas e a correspondente na variável de resposta. Primeiro a teoria principal por trás da técnica é exposta, então a problemática da estimação de efeitos marginais é apresentada e um procedimento computacional de estimação e cálculo dos efeitos marginais baseado em simulações de Monte Carlo é apresentado. Por fim, um estudo de caso com dados de preços de imóveis ilustra uma aplicação do procedimento. 


\textbf{Palavras-chave}: Florestas Aleatórias; Aprendizado de Máquina; Econometria. % Palavra-chave inicia-se com maiúscula
\end{resumo}

% ------------------------------------------------------------------
% No caso do abstract, faremos a mesma coisa. Só adicionaremos a 
% opção abstract como o argumento do comando.
%-------------------------------------------------------------------

\begin{resumo}[ABSTRACT] % ESCREVER EM LETRAS MAIÚSCULAS
	
Random Forest is a supervised learning statistical model built up of decision trees, whose practical applications range from credit risk and asset prices prediction, to the evaluation of public policies, recognition of visual patterns, fraud prevention and disease detection in medical exams. Although widely used and disseminated, there is little literature documenting the systematic relationships between variations in explanatory variables and the corresponding one in the response variable. The theory behind the technique is exposed, then the problem of estimating marginal effects is presented and a computational procedure for estimating and calculating marginal effects based on Monte Carlo simulations is shown. Finally, a case study with real estate data illustrates an application of the procedure.

\textbf{Keywords}: Random Forests; Machine Learning; Econometrics. 	
\end{resumo}



