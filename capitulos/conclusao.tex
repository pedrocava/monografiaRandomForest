\chapter{Considerações Finais}
\label{cap:conclusoes}

Este trabalho se compromete com reprodutibilidade e ciência aberta. Todo o código de estimação de modelos, preparação dos dados, tabelas, imagens e implementação do procedimento está disponível no repositório https://github.com/pedrocava/monografiaRandomForest. 


Foi demonstrado como construir uma floresta aleatória partindo de primeiros princípios. Depois, através de notação matricial clássica no campo, como e o que são modelos lineares. A problemática de estimar efeitos marginais foi apresentada e um procedimento computacionalmente simples para isso em seguida, demonstrando como algumas nuances passavam por cima dos modelos lineares.


Uma limitação não-resolvida da técnica é a inferência dos efeitos marginais computados. É possível distingui-los estatisticamente de zero com algum teste de hipótese? Isso equivale a estabelecer uma ponte entre as duas culturas de \citeonline{breiman2001statistical}. A técnica aqui desenvolvida pode ser estendida para aceitar testes apropriados, aproximando as duas abordagens.


Não parece um problema intratável. Se florestas aleatórias têm distribuição normal e estimamos várias, então temos dados e podemos fazer inferência se conhecermos a variância da floresta. Não é um problema trivial, mas está ao alcance da pesquisa. Essa avenida de pesquisa é provavelmente a de maior interesse ao econometrista aplicado e parte diretamente dos princípios elaborados nesta monografia.





