% ------------------------------------------------------------------
% Exemplo de introdução gerada por textos dummys a partir do
% lipsum
%-------------------------------------------------------------------


\chapter{Introdução}
\label{cap:intro} % faço a referência na bibliografia
\section{Motivação e Estado da Literatura}

O modelo descrito em \citeonline{breiman2001random} apresenta um procedimento de estimação de uma sequência de Árvores de Decisão a partir de subamostras independentes e identicamente distribuídas de um conjunto de dados. O nome da técnica vem do algorítimo consistir em montar uma coleção de Árvores de Decisão e empregar procedimentos de amostragem para que cada árvore individual seja exposta a uma subamostra aleatoriamente encontrada dos dados.

Modelos de Floresta Aleatória (no original, \texit{Random Forest}) têm aplicações em áreas variadas. Pesquisa biomédica, identificação de padrões em visão computacional, finanças e microeconomia aplicada são áreas que já difundiram seu uso. Um entendimento próprio de efeitos marginais é particularmente importante na aplicação econométrica da técnica porque permite a eventual construção de regressão em dois estágios e formas mais precisas de estimar parâmetros em desenhos de Variáveis Instrumentais.

Em revisões de literatura \cite{siroky2009navigating, biau2016random} costuma-se enfatizar alguns resultados sobre a performance da técnica - que entrega capacidade preditiva elevada \cite{gu2018empirical}. Geralmente falando, com o crescimento do número de árvores o primeiro momento da distribuição do erro de previsão fora da amostra converge a zero, seu segundo momento diminui e o custo computacional de efetivamente estimar a floresta cresce com derivada decrescente. O procedimento de estimação original de Breiman fornece um estimador (i) não-viesado, (ii) consistente e (iii) factível, todas propriedades centrais na aplicabilidade dessa técnica. Assim como em modelos lineares, também é possível decompor variância explicada do modelo. \citeonline{stijven2011separating} introduziram algumas ferramentas de regressão simbólica no modelo original de Breiman para explicar importância relativa de regressores e há maneiras de abordar o mesmo problema usando Valores de Shapley, vindos da Teoria dos Jogos Cooperativos \cite{cohen2007feature}.

Avanços recentes na literatura focam em frentes diversas. Uma que dialoga com o problema a ser estudado nesta monografia é como conciliar os modelos com bancos de dados com uma quantidade crescente de variáveis e, portanto, alta dimensionalidade. Uma direção de pesquisa foca em adequar a teoria que fundamenta a técnica ao contexto em alta dimensão \cite{athey2019generalized}, outra em seleção de variáveis e redução de dimensionalidade \cite{hastie2015statistical}. 

Esse esforço de pesquisa gerou um princípio de entendimento melhor de efeitos de tratamento. \citeonline{ikonen2016machine} desenvolveu uma tećnica (rudimentar) de estimação de efeitos de tratamento sintéticos - uma extensão da técnica de Controles Sintéticos \cite{abadie2010synthetic}. A dupla Athey e Wager, particularmente ativa neste debate, deixou algumas contribuições importantes. Notadamente em contexto de alta dimensão \cite{athey2018approximate} e em contexto de tratamentos heterogêneos \cite{wager2018estimation}. 

Em modelos lineares paramétricos sabemos ser possível aproximar não-linearidades arbitrárias imputando no modelo funções não-lineares dos regressores originais, afinal o que caracteriza um modelo estatístico univariado como \textbf{linear} é a variável resposta ser uma função linear do produto interno de dois vetores, o de explicativas e o de parâmetros estimados. Em Florestas Aleatórias sabemos que por desenho não é necessário inserir funções não-lineares dos regressores para acomodar não-linearidades, então essa ponte já está estabelecida, no entanto pouco é sabido sobre o comportamento dos efeitos marginais.

O objetivo desta monografia é propor um procedimento sistemático de estimação de efeitos marginais de tratamento, avaliar o quão (in)constantes são os efeitos marginais e o quão sensíveis aos outros regressores eles são. A metodologia empregada será estimar um modelo de Floresta Aleatória, computar suas previsões na observação mediana dos dados e em observações medianas com perturbações controladas em cada regressor. O resultado é uma curva relacionando valor predito com valores dos regressores, cuja derivada - se computável - entrega os efeitos marginais.


 
 \section{Organização da Monografia}
 
 Veremos como uma árvore de decisão pode distinguir observações e captar não-linearidades no processo gerador. Depois, que agregar árvores pode produzir uma espécie de mitigação de viés que uma árvore individual possa ter, construímos uma maneira de medir o quão bem o processo de voto democrático de cada árvore da floresta discrimina observações com sua Margem e que a partir dela podemos computar uma medida da performance preditiva em dados que não foram apresentados ao modelo em sua fase de treinamento. Por fim, veremos que o erro de generalização com dados desconhecidos tem um limite superior dado um tamanho de floresta. Estes são os fatos básicos do funcionamento do modelo.
 
 
 
 O capítulo 2 apresenta o algorítimo de computação dos efeitos marginais e o desenho da simulação. O capítulo 3 apresenta algumas estatísticas descritivas dos dados e das estimativas de Florestas. O capítulo 4 mostra as simulações e a computação dos efeitos marginais em si. O capítulo 5 conclui com alguns fatos estilizados.


